\input{../configuration}
\usepackage{tikz-3dplot}

\title{ECE 155: Course Summary}

\author{Jeff Zarnett \\ \small \texttt{jzarnett@uwaterloo.ca}}
\institute{Department of Electrical and Computer Engineering \\[-1ex]
  University of Waterloo}
\date{\today}

\begin{document}

\begin{frame}
  \titlepage

  \vfill
  \begin{center}
    \CcGroupBySa{0.83}{0.95ex}\\
                  {\tiny\CcNote{\CcLongnameByNcSa}}
                  \vspace*{-2.5ex}
  \end{center}

\end{frame}


\begin{frame}
\frametitle{Exam Coverage}

The exam is based on material we have covered in the lectures \& labs.

The exam is cumulative: covers from the first lecture to the last.

Lab material you are expected to know covers Labs 0 through Lab 4.

\end{frame}


\begin{frame}
\frametitle{Lectures}

{\LARGE
Lecture 1: Introduction, Sensors \& Actuators

Lecture 2: Introduction to Java

Lecture 3: Java II

Lecture 4: Java III}

\end{frame}

\begin{frame}
\frametitle{Lectures}

{\LARGE

Lecture 5: XML, Android, Eclipse

Lecture 6: Events, Polling, Interrupts

Lecture 7: Android II

}

\end{frame}

\begin{frame}
\frametitle{Lectures}

{\LARGE
 
Lecture 8: Software Design Patterns

Lecture 9: Version Control

Lecture 10: Engineering Design \& Analysis 

}

\end{frame}

\begin{frame}
\frametitle{Lectures}

{\LARGE

Lecture 11: Multithreading

Lecture 12: Dev Planning / Comp. Decision

Lecture 13: Android III

Lecture 14: UML
}

\end{frame}

\begin{frame}
\frametitle{Lectures}

{\LARGE

Lecture 15: Testing / JUnit

Lecture 16: Testing / Coverage

Lecture 17: More on Testing

}

\end{frame}

\begin{frame}
\frametitle{Lectures}

{\LARGE

Lecture 18: Requirements, Specifications

Lecture 19: Debugging

Lecture 20: Debugging II
}

\end{frame}

\begin{frame}
\frametitle{Lectures}
{\LARGE

Lecture 21: Planning

Lecture 22: Debugging III

Lecture 23: Debugging IV
}

\end{frame}

\begin{frame}
\frametitle{Lectures}
{\LARGE

Lecture 24: Software Lifecycle Models

Lecture 25: Code Reviews

Lecture 26: Refactoring

Lecture 27: V\&V, Maintenance
}

\end{frame}


\begin{frame}
\frametitle{Lectures}
{\LARGE

Lecture 28: Timers

Lecture 29: Simulation

Lecture 30: Bricolage, Licenses

Lecture 31: Android IV
}

\end{frame}

\begin{frame}
\frametitle{Lectures}
{\LARGE

Lecture 32: Software Architecture

Lecture 33: Software Communication

Lecture 34: Security

Lecture 35: Library Card
}

\end{frame}



\begin{frame}
\frametitle{The \#1 Most Frequently Asked Question}

{\LARGE
Will we have to write code?!?
}

\end{frame}

\begin{frame}
\frametitle{The \#1 Most Frequently Asked Question}

{\LARGE
Will we have to write code?!? \alert{Yes.}
}

\end{frame}

\begin{frame}
\frametitle{The \#2 Most Frequently Asked Question}

{\LARGE
Will there be more code on the final than there was on the midterm?
}

\end{frame}

\begin{frame}
\frametitle{The \#2 Most Frequently Asked Question}

{\LARGE
Will there be more code on the final than there was on the midterm? \alert{Yes.}
}

\end{frame}


\begin{frame}
\frametitle{Basic Exam Information}

\begin{table}[h]
 \begin{tabular}{|l l|}
        	\hline
			~ & ~ \\	
			Date of Exam: & \textbf{Thursday, 7 August 2014} \\
			Time Period: & \textbf{Start: 09:00. End: 11:30}\\
			Duration of Exam: & \textbf{150 minutes}\\
			Number of Pages (incl. cover): & \textbf{11}\\
			Exam Type: & \textbf{Closed Book}\\
			Additional Materials: & \textbf{No Additional Materials Allowed}\\
			~ & ~\\
			\hline
          \end{tabular}
\end{table}

We are writing in: PAC 7,8 \\
It does not matter where you sit.

\end{frame}


\begin{frame}
\frametitle{Exam Instructions}

\begin{enumerate}
	\item No aids are permitted. No calculators of any type are permitted.
	\item Turn off all communication devices.
	\item There are eight (8) questions, some with multiple parts. Not all are equally difficult.
	\item The exam lasts 150 minutes and there are 120 marks.
	\item If you feel like you need to ask a question, know that the most likely answer is ``Read the Question''. No questions are permitted. If you find that a question requires clarification, proceed by clearly stating any reasonable assumptions necessary to complete the question. If your assumptions are reasonable, they will be taken into account during grading.  
\end{enumerate}

\end{frame}



\begin{frame}
\frametitle{How to Prepare}

How to prepare for the final exam:

\begin{enumerate}
	\item Review lecture notes and slides.
	\item Review the tutorial slides.
	\item Understand your lab solutions.
	\item Try old assignments and old exams.
	\item Ask for extra help if you need it (we have many TAs + 2 instructors).
\end{enumerate}

\end{frame}

\begin{frame}
\frametitle{Exam Tips}

Tips for the Exam:

\begin{enumerate}
	\item Take the time to read the question carefully.
	\item You can use point form instead of full sentences.
	\item Don't leave questions blank - Nothing on the page = 0 marks.
	\item Do the questions you know (or find easy) first, then move on to more challenging ones.
	\item Keep an eye on the time.
	\item Sleep the night before (all nighters are bad).
	
\end{enumerate}

\end{frame}


\begin{frame}
\frametitle{About Grades}

No grades can be released until after the end of exams.

You'll see them in Quest when grades for the term become available.


\end{frame}

\begin{frame}
\frametitle{Tips for Work Term Reports}

You have to write work term reports in the upcoming co-op term.

A subject I know lots about...\\
\quad I wrote 4 reports (we had to do 4 instead of 3).\\
\quad I also marked about 100 of them.

Some basic tips will help you avoid the dreaded resubmission.

\end{frame}


\begin{frame}
\frametitle{Work Report Tips}

Spelling, grammar and formatting count as well.\\
	\quad You can ask someone to proofread, but acknowledge their help.
	
Follow the checklist (Appendix C in the guidelines).

The Contributions section is boring to write, but it's mandatory.

\end{frame}

\begin{frame}
\frametitle{Work Report Tips}

You need technical content. This also means you need numbers.\\
	\quad Describing things in qualitative terms is not enough.

Figures and tables are valuable tools.\\
	\quad ``A picture is worth a thousand words''.

\end{frame}


\begin{frame}
\frametitle{\#1 Reason Why Work Reports Fail}

They look like documentation or story-telling.\\
	\quad Documentation: Clicking the ``report'' button produces a CSV file...\\
	\quad Storytelling: First we did $x$, then we did $y$, and then...

The key question: is there a decision in this report?\\
	\quad No decision-making means a failing grade.
	
You can use a Computational Decision-Making Chart in your report.

\end{frame}


\begin{frame}
\frametitle{The End}

That's all for ECE~155. Thanks for a great term!

I wish you success on the exam.

\end{frame}




\end{document}
