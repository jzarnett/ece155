\input{configuration}

\title{Lecture 25 --- Reviews}

\author{Patrick Lam \& Jeff Zarnett \\ \small \texttt{p.lam@ece.uwaterloo.ca} \& \texttt{jzarnett@uwaterloo.ca}}
\institute{Department of Electrical and Computer Engineering \\
  University of Waterloo}
\date{\today}

\begin{document}

\begin{frame}
  \titlepage
\end{frame}

\part{Reviews}
\frame{\partpage}

\begin{frame}
\frametitle{On Reviews}

\begin{changemargin}{1cm}

\structure{Review}: \\activity where reviewers examine a work product
to provide feedback.\\[1em]

Advantage: \\
reveal defects early---defects less costly to fix.  \\[1em]

What to review:\\
requirements specifications; schedules; bug reports; \\design documents;
\structure{code}; test plans; test cases.

\end{changemargin}

\end{frame}

\begin{frame}
\frametitle{Informal vs Formal}

\large
\begin{changemargin}{1cm}
\begin{itemize}
\item \emph{informal review}: written or verbal review
  requested by a developer of a work product.\\[1em]
\item \emph{formal review}: written review conducted by a team
  leader or a moderator to identify, document,
  and fix defects in a work product.  
\end{itemize}
\end{changemargin}

\end{frame}

\begin{frame}
\frametitle{Types of Reviews}

\begin{changemargin}{1cm}
\begin{itemize}
\item \structure{desk check}: informal review; \\ \qquad author distributes
work to peers for reviews and comments.
\item \structure{walkthrough}: informal review meeting; \\ \qquad moderated by the author.
\item \structure{inspection}: formal review meeting; \\ \qquad guided by a moderator.\\
Produces a log of identified defects in a work product.
\item \structure{code review}: software inspection \\ \qquad identifying, logging, and perhaps correcting bugs.
\end{itemize}
\end{changemargin}

\end{frame}

\begin{frame}

\frametitle{Desk Checks}

\begin{changemargin}{1cm}
\structure{Desk check}: first line of defence against defects.  

\begin{itemize}
\item can speed up formal inspections \\ by taking care of simple defects in
desk checks first. 
\item for many work products, desk checks suffice; \\ often don't need a formal inspection.
\end{itemize}

But:
\begin{itemize}
\item only effective if taken seriously. 
\item easy
to just say ``LGTM'' (Looks Good To Me)\\ without actually checking the
product. \\[1em]

It's important to spend enough time on desk checks, and
managers must allocate time for them. 
\end{itemize}
\end{changemargin}

\end{frame}

\begin{frame}
\frametitle{Walkthroughs}

\begin{changemargin}{1cm}
\structure{Walkthrough}: guided review of a work
product.

\begin{itemize}
\item allow people with less expertise to review a work
product;
\item users of the work product often invited to
walkthroughs.  
\item New points-of-view often help identify defects.
\end{itemize}~\\


Author of the work product presents the design and
ensures that the attendees understand its design.\\[1em]

How a walkthrough works:
\begin{itemize}
\item Before: distribute presentation materials.
\item During: solicit feedback from the audience.
\item After: follow up with attendees who have helped out by giving comments.
\end{itemize}
\end{changemargin}

\end{frame}

\begin{frame}
\frametitle{Inspection}

\begin{changemargin}{1.5cm}
\structure{Inspection}: formal review meeting where participants\\
identify and document defects or possible improvements. \\[1em]

Participants identify, and propose
solutions to, defects. \\[1em]

A good mix  of participants (but not too many!) helps\\
find previously-overlooked defects (subtle, complex bugs).

\end{changemargin}

\end{frame}

\begin{frame}
\frametitle{Steps in a formal inspection}

\begin{changemargin}{1cm}
\begin{itemize}
\item \structure{Preparation}: Before the meeting, distribute the work project to each member of the inspection team, 
plus a checklist indicating what to review.\\[0.5em]
\item \structure{Overview}: Moderator
provides an overview of the item.\\[0.5em]
\item \structure{Page-by-page Review}: Moderator walks the inspection team
through the work product and logs defects.\\[0.5em]
\item \structure{Rework}: Afterwards, the author
goes through the list of defects and fixes them.\\[0.5em]
\item \structure{Follow-up}: Inspection team members verify that
the author has fixed the defects.\\[0.5em]
\item \structure{Approval}: Inspection team approves the work.\\[0.5em]
\end{itemize}
We do something similar for master's and PhD theses.
\end{changemargin}

\end{frame}

\begin{frame}
\frametitle{Code Review}

\begin{changemargin}{1cm}

\structure{Code review}: examines source code (usually a patch)\\
to identify defects or possible improvements.\\[1em]

Coverage options:
\begin{itemize}
\item review everything (Mozilla); or,
\item review a representative sample.
\end{itemize}

Representative sample: \\ \qquad developers tend to repeat the same mistakes.\\
If you find one bug, look for similar ones nearby.
\end{changemargin}
\end{frame}

\begin{frame}
\frametitle{Code Review (sampling)}

\begin{changemargin}{1cm}
When sampling, here are some places to look:
\begin{itemize}
\item source code that only one person has the expertise to maintain;
\item tricky algorithms that are susceptible to defects;
\item source code that calls difficult-to-use libraries;
\item code written by inexperienced developers; and
\item functions that could fail catastrophically if a defect is present.
\end{itemize}
\end{changemargin}
\end{frame}

\begin{frame}
\frametitle{Code Review (who? what?)}

\begin{changemargin}{1cm}
In industry: by team?\\[1em]

Open-source world: 1 or 2 experienced developers, independently.\\[1em]

What to look for:
\begin{itemize}
\item clarity, maintainability, accuracy, reliability,
\item robustness, security, scalability, reusability, efficiency.
\end{itemize}
\end{changemargin}

\end{frame}

\begin{frame}
\frametitle{Pair Programming}

\begin{changemargin}{1cm}
\Large 
Part of Extreme Programming.\\[1em]

Can also serve as instant code review.
\end{changemargin}

\end{frame}

\part{Reviews at University}
\frame{\partpage}

\begin{frame}
\frametitle{Reviews at University}

\begin{changemargin}{1cm}
We have the students do a lot of programming assignments...\\
\quad but we do not review student code.

We do not give feedback on variable names, comments, etc.

Irrelevant to the compiler \& execution,\\
\quad but important when someone (else) will need to read it.

\end{changemargin}
\end{frame}

\begin{frame}
\frametitle{Reviews at University}

\begin{changemargin}{1cm}
UW projects are, at most, 4 months long.\\
\quad Possible exception: 4$^{th}$ year design project.

There are no consequences for writing throwaway code.

Maybe on co-op terms, but how much does it happen?

Trying to ask TAs to do code reviews in ECE~155 labs.

\end{changemargin}
\end{frame}

\begin{frame}
\frametitle{Problem Decomposition}

\begin{changemargin}{1cm}
Key thing to look for in code reviews: problem decomposition.

Take a big, complex problem, break it down into a number of smaller problems that are easier to solve. 

Each subproblem can be broken down further if necessary.

\end{changemargin}
\end{frame}

\begin{frame}
\frametitle{Problem Decomposition}

\begin{changemargin}{1cm}

If your starting problem is ``write ATM software'':

Subproblems: Withdraw cash, deposit cash, check balance. 

Each of those will need to be broken down into some series of other subproblems (like verify card and PIN). 

\end{changemargin}
\end{frame}

\begin{frame}
\frametitle{Reviewing Problem Decomposition}
\begin{changemargin}{1cm}
Good programmers decompose the problems well into subproblems that can be:
\begin{itemize}
 \item clearly described, 
 \item independently implemented, and 
 \item easily tested. 
\end{itemize}
Documentation is often written in advance, but writing the documentation is simple because of the good structure.

\end{changemargin}
\end{frame}

\begin{frame}
\frametitle{Reviewing Problem Decomposition}
\begin{changemargin}{1cm}
Adequate programmers decompose problems reasonably. 

They tend to have some awkward data structures which result in a lot of special-case code. 

The code seems to be ``debugged'' into existence.

Documentation written all at the end once things are finished.

\end{changemargin}
\end{frame}


\begin{frame}
\frametitle{Reviewing Problem Decomposition}
\begin{changemargin}{1cm}
Poor programmers decompose problems seemingly randomly. 

Unhelpful variable and method names like \texttt{x}, \texttt{foo}, or \texttt{doIt()}. 

Code is often poorly tested and fails on boundary conditions. 

It sometimes appears that the code is ``evolved'' into existence: make random changes and see if that improves the output. 

Documentation, if it exists, is difficult to read or misleading.

\end{changemargin}
\end{frame}

\begin{frame}
\frametitle{Reviewing Problem Decomposition}
\begin{changemargin}{1cm}
Problem decomposition is a skill; improve by practicing.

Define your subproblems well, choose appropriate variable names, and write an outline of documentation early on. 


\end{changemargin}
\end{frame}

\begin{frame}
\frametitle{Reviewing Problem Decomposition}
\begin{changemargin}{1cm}
Proper decomposing of the problems is valuable at UW: \\
\quad you can complete assignments quicker and with fewer errors. 

Variable names might not make a difference in assignments. 

It will get you into the correct habits for later\\
\quad and just might impress your employer!

\end{changemargin}
\end{frame}



\part{Reviews for Open-Source Projects}
\frame{\partpage}

\begin{frame}
\frametitle{How Open-Source Projects Work}

\begin{changemargin}{1cm}
Typically: 

\begin{itemize}
\item an official repository. (SVN, Git, Mercurial)\\[1em]
\item a set of committers, who may commit changes.\\[1em]
\end{itemize}

Outside contributors: may send patches \\ \qquad (bug fixes, new features).
\begin{itemize}
\item a committer reviews the patch before committing it.
\end{itemize}

Committers may/must also seek review for their patches.
\end{changemargin}
\end{frame}

\begin{frame}
\frametitle{Case Study: Reviews at Mozilla}

\begin{changemargin}{1cm}
Mozilla Foundation: develops the Firefox web browser \\
(and other
projects). 
\end{changemargin}

\begin{center}
\includegraphics[width=.5\textwidth]{images/firefox-512-noshadow.png}
\end{center}
\end{frame}

\begin{frame}
\frametitle{Case Study: Reviews at Mozilla}

\begin{changemargin}{1cm}
Huge codebase $\Rightarrow$ elaborate
reviewing
policy\footnote{\url{http://www.mozilla.org/hacking/reviewers.html}, 
\url{https://developer.mozilla.org/en/Code_Review_FAQ}}.

\begin{itemize}
\item require at least one review (``owner/peer review'') for all
patches, plus
\item second review (``super-review'') for many 
patches.
\end{itemize}
\end{changemargin}
\end{frame}

\begin{frame}
\frametitle{Case Study: Peer review at Mozilla}
\begin{changemargin}{1cm}
\structure{Owner/peer review}: by a domain expert who understands the code
being modified and the implications of the change.  

\begin{quote} 
A
review is focused on a patch's design, implementation, usefulness in
fixing a stated problem, and fit within its module.
\end{quote}

Reviews check for: 
\begin{itemize}
\item whether the patch fixes a problem; 
\item API/design; 
\item maintainability; 
\item security; 
\item integration; 
\item testing; and
\item license compliance.
\end{itemize}
\end{changemargin}
\end{frame}

\begin{frame}
\frametitle{Case Study: Super-reviews at Mozilla}

\begin{changemargin}{1cm}
Super-reviews by ``strong hackers''.
\begin{itemize}
\item understand the way
Mozilla code is supposed to look,
\item need not have domain expertise.
\end{itemize}
~\\[1em]

They look out
for: 
\begin{itemize}
\item proper use of APIs; 
\item adherence to Mozilla's portability
guidelines; 
\item cross-module effects; and 
\item respect of Mozilla coding
practices.
\end{itemize}

\end{changemargin}
\end{frame}

\begin{frame}
\frametitle{Other organizations}

\begin{center}
\includegraphics[width=.4\textwidth]{images/Googlelogo.png} \quad
\includegraphics[width=.4\textwidth]{images/Tux.png}
\end{center}
\begin{changemargin}{1cm}
Many organizations, including Google and Linux kernel hackers,
review extensively.\\

Gerrit\footnote{\url{http://lwn.net/Articles/359489/}}: a tool out of
Google for code reviews.

\end{changemargin}

\end{frame}


\end{document}
