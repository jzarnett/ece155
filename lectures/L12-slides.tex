\input{configuration}

\title{Lecture 12 --- Computational Decision-Making}

\author{Patrick Lam \\ \small \texttt{p.lam@ece.uwaterloo.ca}}
\institute{Department of Electrical and Computer Engineering \\
  University of Waterloo}
\date{\today}


\begin{document}

\begin{frame}
  \titlepage
\end{frame}

\begin{frame}
\frametitle{Systematic Decision-Making}

\begin{changemargin}{1cm}
General process for making decisions:

\begin{enumerate}
\item List all choices.
\item List all evaluation attributes (criteria).
\item Compare lists and remove impractical choices.
\item Evaluate the advantages and disadvantages of each remaining choice
according to all of the criteria.
\end{enumerate}

But how?
\end{changemargin}

\end{frame}

\begin{frame}
	
	
	\vspace{8em}
	
	\begin{block}
	 {\huge Computational Decision Making}
	\end{block}
	
\end{frame}


\begin{frame}
\frametitle{Computational Decision Making}

\begin{changemargin}{1cm}

Goal: \structure{quantify} the decision-making process.

Assign a \structure{weight} to each criterion \\ \qquad (according to importance); and\\
assign a \structure{score} for each choice.

Combining weights and scores,\\
compute the \structure{value} of
a payoff function for each choice.

\end{changemargin}

\end{frame}

\begin{frame}
\frametitle{CDM: Assigning Weights and Scores}

\begin{changemargin}{1cm}
Assume that there are $m$ choices and $n$ criteria. 
\begin{enumerate}
\item Assign a weight
  $w_j$ to each criterion. \\
  Ensure $\sum_{j=1}^n w_j = 1$ (or
  100\%). \\ Higher weights = more important.
\item For each choice $i$ and criterion $j$, assign score $p_{ij}$,\\
  which summarizes goodness of $i$ with respect to $j$.\\

  \[ p_{ij} \in [0, 1] \]
\end{enumerate}
\end{changemargin}
\end{frame}

\begin{frame}
\frametitle{CDM: Computing Scores and Payoffs}

\begin{changemargin}{1cm}
[Assume that there are $m$ choices and $n$ criteria.]

We can then compute scores $s_{ij}$ and a payoff $f_i$.\\

Choose alternative with
the largest expected payoff $f_i$:
\[ s_{ij} = p_{ij} w_j; \qquad f_i = \sum_{j=1}^n s_{ij} = \sum_{j=1}^n p_{ij} w_j, \]
(i.e. dot product of weights and scores).

\end{changemargin}

\end{frame}

\begin{frame}

\frametitle{Normalizing Scores}

\begin{changemargin}{1cm}

 Rather than choosing scores
$p_{ij} \in [0, 1]$, \\
you can instead normalize the scores:
\begin{enumerate}
\item Assign $c_{ij} \in \mathbb{R}$ for each alternative $i$ and criterion $j$.
Use \emph{same units} for all alternatives of a 
 criterion. \\ You can use different units for different criteria.
\item For each criterion, calculate
\[ C_j = \max \{ |c_{1j}|, |c_{2j}|, \ldots, |c_{mj}| \}, \]
so that
\[ p_{ij} = \frac{c_{ij}}{C_j}. \]
\item The payoff function is then given by:
\[ f_i = \sum_{j=1}^n p_{ij} w_j = \sum_{j=1}^n \frac{c_{ij}}{C_j}  w_j. \]
\end{enumerate}

\end{changemargin}

\end{frame}

\begin{frame}
\frametitle{CDM: Worked Example}

\begin{changemargin}{1cm}
``How should I get to Montreal?''

Done on the board to facilitate understanding.

\end{changemargin}

\end{frame}

\begin{frame}
\frametitle{CDM: Potential Flaws in Example}

\begin{changemargin}{1cm}

Did we get an optimal decision? \\

Let's look at threats to validity:

\begin{itemize}
\item (not an issue here) scores might be stuck in a
small subrange of the possible range, affecting values.
%\item The values for each criteria have different ranges: flexibility
%is in $[0, 0.5]$ while cost is in $[0.5, 1.0]$ and time is in $[0.45, 1.0]$.
\item values for flexibility are subjective:\\
how you assign numbers to options changes  outcome.
\item cost depends on what you include; \\
for instance, gas alone would be \$72.50; \\
but it would be less if you carpooled.
\item weights are subjective.
\end{itemize}

\end{changemargin}

\end{frame}

\begin{frame}
\frametitle{Sensitivity Analysis}

\begin{changemargin}{1cm}
Investigate impact of potential flaws.

\structure{Sensitivity Analysis}: study of how variations
influence outcome of a mathematical model.\\[1.5em]

Poor person's sensitivity analysis: do what-if
calculations.

Replace the original value with some other value\\
and repeat the analysis. \\

(Possible issues if scores not independent.)

\end{changemargin}

\end{frame}

\begin{frame}
\frametitle{Sensitivity Analysis: Examples}

\begin{changemargin}{1cm}
Train used to always be late, \\ so you might assign 9
instead of 7.7. \\ \qquad (mostly on time recently.) 

Or, you
might want to get to the airport slightly earlier than me, so you might also
change the time for plane to 5.

Or, you might be more cost-sensitive and less time-sensitive.

Do these change the best decision?
\end{changemargin}
\end{frame}

\begin{frame}
\frametitle{CDM: Other References}

\begin{changemargin}{1em}
Previous notes contained an example (``Where should I go to
University?'') and a reference to \emph{Introduction to
  Professional Engineering in Canada}, Example 15.3
(pp. 232--233). 

Another example: Waterloo Region's
Rapid Transit alternatives \\ \qquad (no actual numbers). 

Another example: choosing which phones to buy for ECE155.
\end{changemargin}
\end{frame}

\begin{frame}
\frametitle{Discussion}

\begin{changemargin}{2em}
Computational decision-making systematically includes diverse factors
to arrive at a ``best'' decision.

Disadvantages:
\begin{itemize}
\item must numerically estimate weights and scores;
\item estimating may be infeasible;
\item estimates may reflect pre-conceived notions;
\item does not generate new alternatives.
\end{itemize}~\\

Advantage:
\begin{itemize}
\item ensure that you give each alternative consideration.
\end{itemize}~\\

Note: you're not required to pick the highest-scoring alternative.

\end{changemargin}
\end{frame}



\end{document}
