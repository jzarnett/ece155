\input{configuration}

\title{Lecture 30 --- Software Bricolage}

\author{Patrick Lam \& Jeff Zarnett\\ \small \texttt{p.lam@ece.uwaterloo.ca} \& \texttt{jzarnett@uwaterloo.ca}}
\institute{Department of Electrical and Computer Engineering \\
  University of Waterloo}
\date{\today}

\begin{document}

\begin{frame}
  \titlepage
\end{frame}

\part{Software Bricolage}
\frame{\partpage}

\begin{frame}
\frametitle{Software Bricolage}
\begin{changemargin}{1cm}
\Large

Today: Assignments vs real-world programming.\\[1em]

Most of your work for school is not open-ended until FYDP.\\[1em]

Some of your work in co-op will be open-ended. We'll see techniques 
for doing this work.
\end{changemargin}

\end{frame}

\begin{frame}
\frametitle{Schoolwork}

\begin{center}
\includegraphics[width=.4\textwidth]{images/1502_cookies}
\end{center}

\begin{changemargin}{1cm}
Ideally:
\begin{itemize}
\item well-specified
\item we provide tools and libraries you'll need (e.g.~{\tt LineGraphView}).
\end{itemize}
\end{changemargin}

\end{frame}

\begin{frame}
\frametitle{More on schoolwork}

\begin{changemargin}{1cm}
We have educational goals, so you get:
\begin{itemize}
\item Lots of template code---you fill in the blanks.
\item Problems you can solve cleanly in a single language (Java).
\end{itemize}

e.g. Assignment 4: less than 50 lines; my Lab 1: 110 lines.\\[1em]

By the way, if you want to do an open-ended project for Labs 3 and 4,
talk to me.

\end{changemargin}

\end{frame}

\begin{frame}
\frametitle{Real-world Programming}

\Large
\begin{changemargin}{1cm}
Often: check out large codebase, \\ \qquad fix~something.\\[1em]

Sometimes: start a project from scratch.\\
\qquad (But not really from scratch).
\end{changemargin}
\end{frame}

\begin{frame}
\frametitle{Getting Started}

\begin{changemargin}{1cm}
\Large
You have a goal.\\[1em]

Need to formalize the goal:
\begin{itemize}
\item even high-quality software is no good unless it meets requirements.
\end{itemize}
~\\[1em]
ECE451 is all about requirements.
\end{changemargin}

\end{frame}

\begin{frame}
\frametitle{Steps to Build Software (per Philip Guo)}

\begin{changemargin}{1cm}
\Huge
\begin{enumerate}
\item Forage
\item Tinker
\item Weld
\item Grow
\item Doubt
\item Refactor
\end{enumerate}
\end{changemargin}

\end{frame}

\begin{frame}

\frametitle{Step 1: Foraging}

\begin{center}
\includegraphics[width=.4\textwidth]{images/0673_foraging}
\end{center}
{\tiny \hfill (P. Lam collection)}

\begin{changemargin}{1cm}
Look for suitable components/libraries.\\
\qquad (maybe yours, maybe others').\\
\qquad Know what's out there.\\[1em]

It may be documented (if you're lucky).\\[1em]

Components may be in different languages.

\end{changemargin}

\end{frame}

\begin{frame}

\frametitle{Step 2: Tinker}

\begin{tabular}{cc}
\begin{minipage}{.35 \textwidth}
\includegraphics[width=\textwidth]{images/home_hardware}\\
{\tiny \hspace*{.5\textwidth} (P. Lam collection)}
\end{minipage}&
\begin{minipage}{.6\textwidth}
\begin{itemize}
\item What can your code actually do?
\item Experiment with the software! 
\item Give it test inputs.
\item Instrument the code. Modify it.
\end{itemize}
This is very much like debugging.\\[1em]

Repeat foraging and tinkering as needed.
\end{minipage}
\end{tabular}

\end{frame}

\begin{frame}
\frametitle{Step 3: Weld}
\begin{changemargin}{1cm}
\Large
Two potential problems:
\begin{itemize}
\item dependencies;
\item impedence mismatches.
\end{itemize}
\end{changemargin}
\end{frame}

\begin{frame}
\frametitle{Dependencies}

\begin{center}
\includegraphics[width=.8\textwidth]{images/dependencies}
\end{center}

\begin{changemargin}{1cm}
\Large
Sometimes you can't get version you need.\\[1em]
Sometimes required versions conflict.
\end{changemargin}

\end{frame}

\begin{frame}
\frametitle{Impedence mismatches}

\begin{center}
\includegraphics[width=.7\textwidth]{images/shim}
\end{center}

\begin{changemargin}{1cm}
\Large
\item You may have to build a shim\\ \qquad (e.g. XML output, CSV input!)
\end{changemargin}

\end{frame}

\begin{frame}
\frametitle{Step 4: Grow}
\begin{changemargin}{1cm}
\Large
Start building code.\\[1em]
Begin with simple examples, concrete code.\\[1em]
What's the simplest thing that can work?\\[1em]
Challenge: fix bad welds.
\end{changemargin}
\end{frame}

\begin{frame}
\frametitle{Step 5: Doubt}

\Large
\begin{changemargin}{1cm}
\begin{itemize}
\item Don't reinvent the wheel.\\
 Know what's in libraries.\\
 Ask the authors.\\
 Contribute to the library.
\end{itemize}
\end{changemargin}

\end{frame}

\begin{frame}
\frametitle{Step 6: Refactor}

\Large
\begin{changemargin}{1cm}
Clean your code, make it more general.\\[1em]
Improve interactions between your code and others.
\end{changemargin}

\end{frame}

\begin{frame}
\frametitle{Iterate}

\begin{changemargin}{1cm}
\Large
Iterate steps 4--6 as needed.\\
Grow, doubt, refactor.
\end{changemargin}
\end{frame}

\begin{frame}
\frametitle{Using the Web for Programming}

\begin{changemargin}{1cm}
Beware: Don't indiscriminately copy code from the Internet.\\
Policy 71, and lawsuits (in industry).\\[1em]

Highly useful when used properly.
\end{changemargin}

\end{frame}

\begin{frame}
\frametitle{Three Main Ways}

\Large
\begin{changemargin}{1cm}
\begin{itemize}
\item Learn concepts.
\item Clarify existing knowledge.
\item Remind of details.
\end{itemize}
\end{changemargin}
\end{frame}

\begin{frame}
\frametitle{Learning concepts}

\Large
\begin{changemargin}{1cm}
\item Read tutorials.\\
Slow; hard to find good ones.\\
Gives an understanding of how things work.
\item Experiment with sample code.
\end{changemargin}
\end{frame}


\begin{frame}
\frametitle{Clarify existing knowledge}

\begin{changemargin}{1cm}
\Large
\begin{itemize}
\item Have some existing knowledge.
\item Not quite sure about it.
\item Also look up error messages (stackoverflow).
\end{itemize}
\end{changemargin}

\end{frame}

\begin{frame}
\frametitle{Remind of details}

\Large
\begin{changemargin}{1cm}
\begin{itemize}
\item especially syntax: not that important.
\end{itemize}

General tip: refine your queries iteratively.
\end{changemargin}
\end{frame}

\part{Licensing}
\frame{\partpage}

\begin{frame}
\frametitle{Software Licenses}

\begin{changemargin}{1cm}
A \alert{Software License} is a legal instrument that tells us how a piece of software may be used or distributed.

It grants you the rights to do things that would otherwise be an infringement (violation) of copyright law.


\end{changemargin}
\end{frame}

\begin{frame}
\frametitle{Software Licenses}

\begin{changemargin}{1cm}

Answers questions like:

\begin{itemize}
	\item Where, how \& how often can you install the program?
	\item Can you copy, modify, or redistribute it?
	\item Can you look at the underlying source code?
\end{itemize}

\end{changemargin}
\end{frame}


\begin{frame}
\frametitle{Software Licenses: Ignore Them?}

\begin{changemargin}{1cm}

This is boring. Who cares?! We do - can't ignore it.

If you do not explicitly declare a license, you effectively get one anyway. 

Code, like other works, is automatically copyrighted by default.

People can read code, but they have no legal right to use it.
\end{changemargin}
\end{frame}

\begin{frame}
\frametitle{Software Licenses: Proprietary}

\begin{changemargin}{1cm}
Impossible to generalize.

You get the rights specified in the license and that's it.

\end{changemargin}
\end{frame}

\begin{frame}
\frametitle{Software Licenses: GNU GPL}

\begin{changemargin}{1cm}
Most common example of an open-source license.

Referred to as ``Copyleft''.

Ensures code can be used, copied, modified, and redistributed.

Code under this license can't be used in proprietary programs.

Any changes you make must be licensed under GPL.

\end{changemargin}
\end{frame}


\begin{frame}
\frametitle{Software Licenses: GNU GPLv2 \& GPLv3}

\begin{changemargin}{1cm}
More recent update to the GPL: version 3.

Intended to address a shortcoming called \alert{Tivoization}.

Named after the Tivo digital video recorder.

\end{changemargin}
\end{frame}

\begin{frame}
\frametitle{Software Licenses: GNU GPLv2 \& GPLv3}
\begin{center}
\includegraphics[width=0.3\textwidth]{images/tivo.png}
\end{center}

Tivo was based on Linux, open source software.
 
Added hardware protection to prevent modifying the software.

\end{frame}

\begin{frame}
\frametitle{Software Licenses: GNU LGPL}

\begin{changemargin}{1cm}
The Lesser GPL is intended for software libraries.

Binary (compiled) code can be linked to proprietary programs.

Library code must follow GPL restrictions.

\end{changemargin}
\end{frame}


\begin{frame}
\frametitle{Software Licenses: BSD}

\begin{changemargin}{1cm}
A permissive open-source license.

Allows any use of code, even making it a proprietary product.

Disclaims all liability.

No requirement to share changes under any license.

\end{changemargin}
\end{frame}

\begin{frame}
\frametitle{Software Licenses: Mozilla}

\begin{changemargin}{1cm}
Open source license from the people who make Firefox.

Allows source code to be mixed - open-source + proprietary.

Any file licensed under MPL must be open source;\\
\quad But not all files must be open-sourced.

Hybrid between BSD and GPL licenses.

\end{changemargin}
\end{frame}

\begin{frame}
\frametitle{Software Licenses: Public Domain}

\begin{changemargin}{1cm}
No copyright owner - may be used by anyone for any purpose.

All copyrighted works will eventually enter the public domain.\\
\quad\quad But this can take a long time. Life + 70 years.

Authors can explicitly put code in the public domain.

\end{changemargin}
\end{frame}

\begin{frame}
\frametitle{Software Licenses: Public Domain}

\begin{changemargin}{1cm}
In the USA the amount of time before work enters the public domain keeps getting extended.

Disney is a big advocate: want to keep Steamboat Willie (first Mickey Mouse animation) from entering the public domain.

\end{changemargin}
\end{frame}

\begin{frame}
\frametitle{Software Licenses: Public Domain}

\begin{changemargin}{1cm}

\begin{center}
\includegraphics[width=0.75\textwidth]{images/copyright-term.png}
\end{center}

You may safely assume a work will never enter the public domain unless the author chooses to put it there.

\end{changemargin}
\end{frame}


\begin{frame}
\frametitle{License Violations}

\begin{changemargin}{1cm}

License violations are taken seriously and lead to legal action.

Importing open source code into your software can get you into a lot of trouble.

\end{changemargin}
\end{frame}



\begin{frame}
\frametitle{License Violation Example}

\begin{changemargin}{1cm}

Landgericht Hamburg found a company, FANTEC, guilty of violating the GPL in their media player.

They distributed their firmware, containing some software licensed under the GPL (\texttt{iptables}).

They did not distribute the code as the GPL requires.

The court required FANTEC to pay penalty fee \& legal costs.


\end{changemargin}
\end{frame}


\end{document}
